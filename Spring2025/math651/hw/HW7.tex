\documentclass[minion]{homework651}
%% Feel free to add your own commonly used commands to this file.

\newcommand{\Reals}{\ensuremath{\mathbb R}}% Gives you a shortcut for writing the blackboard R for the real numbers - \RR
\newcommand{\Nats}{\ensuremath{\mathbb N}} % Gives you a shortcut for writing the blackboard N for the natural numbers - \NN
\newcommand{\Ints}{\ensuremath{\mathbb Z}} % Gives you a shortcut for writing the blackboard Z for the integer numbers - \ZZ
\newcommand{\Rats}{\ensuremath{\mathbb Q}} % Gives you a shortcut for writing the blackboard Q for the rational numbers - \QQ
\newcommand{\Cplx}{\ensuremath{\mathbb C}} % Gives you a shortcut for writing the blackboard C for the complex numbers - \CC

% Make better absolute value bars and the norm symbol
\newcommand{\abs}[1]{\left|#1\right|}
\newcommand{\norm}[1]{\left|\left|\,#1\,\right|\right|}

%% Now make some equivalents that some people may prefer.
\let\RR\Reals
\let\NN\Nats
\let\II\Ints
\let\CC\Cplx
\let\ZZ\Ints

%Add a shortcut for \rightarrow
\let\ra\rightarrow

%Add a \diam command for diameter
\newcommand{\diam}{\text{diam}}


\def\calB{\mathcal{B}}
\DeclareMathOperator{\Int}{\mathrm{Int}}

\doclabel{Math F651: Homework 7}
\docdate{Due: March 19, 2025}
%\docauthor{Your Name Here}

\begin{document}

Note: The book has Exercises, which are interspersed among the
prose, and Problems, which appear at the ends of the chapters.
It can be easy to confuse the two.  Exercises are denoted in blue.


\begin{problems}

\problem Problem 4-11 b) (Just the connected part; the path connected is similar)

\problem Problem 4-9 [Modified]
Let $M$ be an $n$-manifold.
\begin{subproblems}
\item Show that each component of $M$ is a (connected) manifold.
\item Show that there are at most countably many components.
\item Suppose $f:M\to Z$ is a map into a topoogical space $Z$. Show
that $f$ is continuous if and only if its restriction to each component is.
\item Read Theorem 3.41.  Then conclude that an $n$-manifold is homeomorphic
to a disjoint union of countably many connected $n$-manifolds.
\end{subproblems}

\problem Let $f:X\rightarrow Y$ where $X$ is a space and $Y$ is compact and Hausdorff.  Show that
$f$ is continuous if and only if the graph of $f$ is closed in $X\times Y$.  The graph
of $f$ is $G_f=\{(x,f(x)):x\in X\}$.

\problem If $(X,d)$ is a metric space, a function $f:X\rightarrow X$ is an isometry if
for all $x,y\in X$, $d(f(x),f(y))=d(x,y)$.  Show that every isometry is continuous and injective.
Then show that if  $X$ is compact and $f$ is an isometry 
then $f$ is surjective  as well and quickly conclude that $f$ is a homeomorphism. Hint:
Show that $a$ is not in the image of $f$, then for some $\epsilon>0$, $B_\epsilon(a)$
is also not in the image of $f$.  Then show that if $x_0=a$, $x_1=f(x_0)$, etc, then 
$d(x_n,x_m)>\epsilon$ for $n\neq m$.

\problem Show that if $p$ and $q$ are elements of the interior of the closed unit ball 
$$
\mathbb B^n=\{x\in \Reals^n:|x|\le 1\},
$$
then there is a homeomorphism $\phi:\mathbb B^n\ra \mathbb B^n$ such that $\phi(p)=q$ and such that $\phi(x)=x$ for all $x$
with $|x|=1$.  Be as rigorous as you can, but avoid writing a tome.

\problem 3.22a You'll need to read the material on pages 78--80 first.

\end{problems}
\end{document}