\documentclass[minion]{homework651}
%% Feel free to add your own commonly used commands to this file.

\newcommand{\Reals}{\ensuremath{\mathbb R}}% Gives you a shortcut for writing the blackboard R for the real numbers - \RR
\newcommand{\Nats}{\ensuremath{\mathbb N}} % Gives you a shortcut for writing the blackboard N for the natural numbers - \NN
\newcommand{\Ints}{\ensuremath{\mathbb Z}} % Gives you a shortcut for writing the blackboard Z for the integer numbers - \ZZ
\newcommand{\Rats}{\ensuremath{\mathbb Q}} % Gives you a shortcut for writing the blackboard Q for the rational numbers - \QQ
\newcommand{\Cplx}{\ensuremath{\mathbb C}} % Gives you a shortcut for writing the blackboard C for the complex numbers - \CC

% Make better absolute value bars and the norm symbol
\newcommand{\abs}[1]{\left|#1\right|}
\newcommand{\norm}[1]{\left|\left|\,#1\,\right|\right|}

%% Now make some equivalents that some people may prefer.
\let\RR\Reals
\let\NN\Nats
\let\II\Ints
\let\CC\Cplx
\let\ZZ\Ints

%Add a shortcut for \rightarrow
\let\ra\rightarrow

%Add a \diam command for diameter
\newcommand{\diam}{\text{diam}}


% The following commands set up the material that appears
% in the header.
\doclabel{Math 651: Demo Homework}
\docauthor{Your name here.}
\docdate{January 17, 2022}

\begin{document}
\begin{problems}

\problem Exercise 0.1 \solver{John Gimbel}
If $a$ and $b$ are even integers, then so is $a+b$.
\solution
Let $a$ and $b$ be even integers.  Then there exist integers
$j$ and $k$ such that $a=2j$ and $b=2k$.  But then
\begin{equation}
a+b = 2j+ 2k = 2(j+k).
\end{equation}
Since $j+k\in\Ints$, $a+b$ is even.

\problem Exercise 0.2 \solver{Jill Faudree}
Let $X$ be a set.
\begin{subproblems}
\item An intersection of topologies on $X$ is a topology on $X$.
\item A union of topologies on $X$ need not be a topology.
\end{subproblems}
\subsol
Let $\{\tau_\alpha\}$ be a family of topologies and let $\tau=\cap_\alpha \tau_\alpha$.  
Observe that $\emptyset$ and $X$ belong to $\tau$ as they belong to each $\tau_\alpha$.

Suppose $\{U_\beta\}$ is a family of sets in $\tau$ and let $U=\cup_\beta U_\beta$. 
Fix $\alpha$ and observe that each $U_\beta\in \tau_\alpha$. Since $\tau_\alpha$
is a topology, $U\in\tau_\alpha$.  Since $\alpha$ is arbitrary, $U\in\cap\tau_\alpha=\tau$.

The proof that a finite intersection of sets in $\tau$ belongs to $\tau$ is essentially similar.

\subsol
Let $X=\{1,2,3\}$.  Let $\tau_1 = \{\emptyset, \{1\}, X\}$ and let $\tau_2 = \{\emptyset, \{2\}, X\}$.
It is easy to see that these are topologies.
Let $T=\tau_1\cup \tau_2 = \{\emptyset, \{1\}, \{2\}, X\}$.   Observe that $T$ is not closed under
taking unions as $\{1\}$ and $\{2\}$ are elements of $T$ but $\{1,2\}$ is not.

\problem Exercise 0.3 \solver{Elizabeth Allman}
Let $X$ be a metric space.  Showt that 
the collection of open balls in $X$ forms the basis of a topology.

\solution
We start with a technical lemma.
\begin{lemma}{A}\label{lem:DAMrefine}  
Suppose $B_1=B_{r_1}(x_1)$ and $B_2=B_{r_2}(x_2)$ are 
open balls in $X$ an $x_3\in B_1\cap B_2$.  Then there is an $r>0$ such
that $B_r(x_3)\subseteq B_1\cap B_2$.
\end{lemma}
\begin{proof}
Let $r = \min(r_1-d(x_3,x_1),r_2-d(x-2,x_2)$ and observe that $r>0$.  Now suppose
$z\in B_{r}(x_3)$.  The triangle inequality implies
\begin{align*} 
d(x_1,z)&\le d(x_1,x_3) + d(x_3,z) \\
&< d(x_1,x_3) + r \\
&\le d(x_1,x_3) + ( r_1-d(x_3,x_1) ) \\
&= r_1
\end{align*}
Hence $z\in B_{r_1}(x_1)=x_1$.  Similarly $z\in B_2$, and hence $B_r(z)\subseteq B_1\cap B_2$.
\end{proof}

Let $\mathcal{B}$ be the collection of open balls in $X$.  Fix $x\in X$ and note
that $\cup_{r>0} B_r(x)=X$.  Hence $\mathcal{B}$ covers $X$.  Moreover, by Lemma
\ref{lem:DAMrefine}, $\mathcal{B}$ satisfies the refinement property.  Hence
by the topology construction lemma, $\mathcal{B}$ generates a topology on $X$,
and the open sets are simply the unions of open balls.

\end{problems}
\end{document}