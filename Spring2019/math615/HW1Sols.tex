\documentclass[minion]{homework}
\newcommand{\Reals}{\mathbb{R}}
\doclabel{Math F615: Homework 1 Solutions}
\docdate{January 23, 2019}

\begin{document}

\begin{problems}

\problem Refresh your memory about the statement of Taylor's Theorem
with remainder.  (Appendix A.2 in your text will be helpful).  Then
use it to estimate the number of terms needed for the Taylor polynomial $p(x)$ for $\sin(x)$ such that
\[
|p(x)-\sin(x)|< 10^{-4}
\]
for all $x$ in $[0,\pi]$.  This will require estimating the size
of the remainder term, and you are welcome to use a computer or calculator to assist in this computation.  Then generate a graph
showing the difference between $p(x)$ and $\sin(x)$ on
the interval.  What is the maximum error you actually observe?
\solution
From Taylor's Theorem,
\[
\sin(h) = \sin(0+h) = p(h) + \frac{\sin^{(k+1)}(c)}{(k+1)!}h^{k+1}
\]
where $p(x)$ is the $k^{\rm th}$ order Taylor polynomial, and where
$c$ lies between $0$ and $h$.  For $h$ in $[0,\pi]$ we estimate
\[
\left| \frac{\sin^{(k+1)}(c)}{(k+1)!}h^{k+1} \right| \le \frac{\pi^{k+1}}{(k+1)!}
\]
Let $E_k = \pi^{k+1}/ (k+1)!$.  A numerical computation shows that
$E_{13}\approx 10^{-3}$ but $E_{14}\approx 2\times 10^{-5}$.  

The Jupyter notebook shows a graph of the error in using the order 14 
Taylor polynomial and confirms that the error is indeed less than
$10^4$.

\problem  Solve the initial value problem
\[
y''-y'-6y = 0
\]
for $y(t)$ subject to the initial condition $y(0)=0$ and
$y'(0)=1$.

Then find the general solution of
\[
y''-y'-6y = 1 + t.
\]
\subsol
The characteristic polynomial of the ODE is $\lambda^2-\lambda-6$ which has roots $\lambda_+=3$ and $\lambda_-=-2$.  Thus the
general solution is
\[
y = A e^{3t} + B e^{-2t}
\]
for constants $A$ and $B$.  From the initial condition we determine
$A+B=0$ and $3A-2B=1$ so $A=1/5$ and $B=1/5$.

\subsol
For the general solution, it is enough to find a single solution
of the ODE, and then add on the general solution of the homogeneous
version (which we already solved in the previous problem).

Let us guess a solution of the form $c_1 t + c0$.  
Substituting into the ODE we desire
\[
-c_1 -6(c_1 t+c0)  = 1+t
\]
and therefore $c_1=-1/6$ and $c_0=-5/36$.  The general solution
is
\[
y = A e^{3t} + B e^{-2t} -\frac{t}{6} -\frac{5}{36}.
\]

\problem Implement the bisection algorithm for finding roots.  (You
can look at the Wikipedia entry for a reminder of how the algorithm works.)
Your code should take as its arguments:
\begin{enumerate}
	\item A function $f$.  You will be solving $f(x)=0$.
	\item Two numbers $a$ and $b$.  The desired root should be in $[a,b]$.
	\item A tolerance $\epsilon$.  The approximate root should be within distance $\epsilon$ of the true root.
\end{enumerate}
It should return the approximate root.

Hand in both your code and a session showing its results in
approximating $\sqrt{2}$ to within $10^{-8}$.

\solution

See Jupyter notebook.

\problem Implement Newton's method. 
Your code should take as its arguments:
\begin{enumerate}
	\item A function $f$.  You will be solving $f(x)=0$.
	\item A function $f'$.  This is the derivative of the function $f$.
	\item A number $x_0$, which is the initial approximate root.
	\item A tolerance $\epsilon$.  The approximate root should be be returned when two subsequent iterations of Newton's method yield
	approximations within $\epsilon$ of each other.
\end{enumerate}
It should return the approximate value of the root and the number of
iterations required to find it.

Newton's method can be finicky.  Your code should handle gracefully
error conditions that can occur.

Hand in both your code, and a session showing its results 
approximating $\sqrt{2}$ to within $10^{-12}$ starting from an initial approximation $x_0=2$.

\solution

See Jupyter notebook.
\end{problems}
\end{document}