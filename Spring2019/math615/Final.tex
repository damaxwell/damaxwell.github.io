\documentclass[minion]{homework}
\newcommand{\Reals}{\mathbb{R}}
\doclabel{Math F615: Take-Home Final}
\docdate{Due: May 5, 2019}
\usepackage{graphicx}
\usepackage{enumerate}
\newcommand{\bfx}{\mathbf{x}}
\newcommand{\bfv}{\mathbf{v}}

\begin{document}

{\bf  Rules and format:}
\begin{itemize}

\item You are welcome to discuss this exam with me (David Maxwell) to ask for hints and so forth.
\item  If you find a suspected typo, please contact me as soon as possible and I will
communicate it to the class if needed.
\item You my not discuss the exam with anyone else until after the due date/time.
\item You are permitted to reference any text you would like in solving these problems.
\item Each problem is weighted equally.
\item The due date/time is absolutely firm.
\end{itemize}


\begin{problems}

\problem Text, 1.15

\problem Text, 5.4.  Then, taking $\ell=1$, find the exact solution with
$u(x,0)=\sin(\pi x)$. Finally, write a code that implements the method
and verifies against this initial condition that the desired rate
of convergence is acheived.

\problem Consider the problem
\[
u_{xx} + \gamma u^4 = f(x)
\]
on the interval $0\le x\le 1$ with $u(0)=u(1)=0$.
\begin{subproblems}
\item If $u(x)=\sin(3\pi x)$, what is the value of $f(x)$?
\item Write a code to solve this problem based on the
following approach:
\begin{enumerate}
	\item Use centered differences to as in Section 2.2 of
	your text to approximate the second derivative and
	derive an algebraic system to solve for a vector of unknowns
	$u_i$ that approximate $u(x_i)$. 
	\item Implement a numerical method to solve the system
	(with user-supplied right-hand side $f$ and constant $\gamma$)
	by applying Newton's method to approximate the solution
	of the algebraic system.  Newton's method will be applied
	to a system of the form $F(u)=0$, and iterations should
	stop when the residual norm $||F(u)||$ has been reduced by $10^{-9}$
	of its original value.
	\item Show that with the verification case from part a), and separately with
	$\gamma=0$, that your code exhibits $O(h^2)$ convergence.
\end{enumerate}
\end{subproblems}

\problem The TR-BDF2 method is an implicit second-order
Runge-Kutta method of the following form.
\begin{equation}
\begin{aligned}
u_* &= u_n + \frac{k}{4}\left[ f(u_n) + f(u_*)\right] \\
u_{n+1} &=  \frac{1}{3}\left[4u_*-u_n + kf(u_{n+1}\right]
\end{aligned}
\end{equation}
\begin{subproblems}
\item Show that this method is $L$-stable.
\item Write a numerical code using TR-BDF2 as the basis 
for solving the heat equation $u_t=u_{xx}$ in 
a Method of Lines approach.  Verify, using the test case of
Homework 6, problem 1c, that you observe $O(h^2)$ convergence.
\item Discuss the merits of this strategy versus Backwards Euler
and Crank Nicolson.
\end{subproblems}
\end{problems}
\end{document}