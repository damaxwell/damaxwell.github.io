%%%%%%%%%%%%%%%%%%%%%%%%%%%%%%%%%%%%%%%%%%%%%%%%%%%%%%%%%%%%%%%%%%%%%%%%%%%%%%%%%%%%%%%
%%%%%%%%%%%%%%%%%%%%%%%%%%%%%%%%%%%%%%%%%%%%%%%%%%%%%%%%%%%%%%%%%%%%%%%%%%%%%%%%%%%%%%%
% 
% This top part of the document is called the 'preamble'.  Modify it with caution!
%
% The real document starts below where it says 'The main document starts here'.

\documentclass[12pt]{article}

\usepackage{amssymb,amsmath,amsthm}
\usepackage[top=1in, bottom=1in, left=1.25in, right=1.25in]{geometry}
\usepackage{fancyhdr}
\usepackage{enumerate}

% Comment the following line to use TeX's default font of Computer Modern.
\usepackage{times,txfonts}

\newtheoremstyle{homework}% name of the style to be used
  {18pt}% measure of space to leave above the theorem. E.g.: 3pt
  {12pt}% measure of space to leave below the theorem. E.g.: 3pt
  {}% name of font to use in the body of the theorem
  {}% measure of space to indent
  {\bfseries}% name of head font
  {:}% punctuation between head and body
  {2ex}% space after theorem head; " " = normal interword space
  {}% Manually specify head
\theoremstyle{homework} 

% Set up an Exercise environment and a Solution label.
\newtheorem*{exercisecore}{Exercise \@currentlabel}
\newenvironment{exercise}[1]
{\def\@currentlabel{#1}\exercisecore}
{\endexercisecore}

\newcommand{\localhead}[1]{\par\smallskip\noindent\textbf{#1}\nobreak\\}%
\newcommand\solution{\localhead{Solution:}}

%%%%%%%%%%%%%%%%%%%%%%%%%%%%%%%%%%%%%%%%%%%%%%%%%%%%%%%%%%%%%%%%%%%%%%%%
%
% Stuff for getting the name/document date/title across the header
\makeatletter
\RequirePackage{fancyhdr}
\pagestyle{fancy}
\fancyfoot[C]{\ifnum \value{page} > 1\relax\thepage\fi}
\fancyhead[L]{\ifx\@doclabel\@empty\else\@doclabel\fi}
\fancyhead[C]{\ifx\@docdate\@empty\else\@docdate\fi}
\fancyhead[R]{\ifx\@docauthor\@empty\else\@docauthor\fi}
\headheight 15pt

\def\doclabel#1{\gdef\@doclabel{#1}}
\doclabel{Use {\tt\textbackslash doclabel\{MY LABEL\}}.}
\def\docdate#1{\gdef\@docdate{#1}}
\docdate{Use {\tt\textbackslash docdate\{MY DATE\}}.}
\def\docauthor#1{\gdef\@docauthor{#1}}
\docauthor{Use {\tt\textbackslash docauthor\{MY NAME\}}.}
\makeatother

% Shortcuts for blackboard bold number sets (reals, integers, etc.)
\newcommand{\Reals}{\ensuremath{\mathbb R}}
\newcommand{\Nats}{\ensuremath{\mathbb N}}
\newcommand{\Ints}{\ensuremath{\mathbb Z}}
\newcommand{\Rats}{\ensuremath{\mathbb Q}}
\newcommand{\Cplx}{\ensuremath{\mathbb C}}
%% Some equivalents that some people may prefer.
\let\RR\Reals
\let\NN\Nats
\let\II\Ints
\let\CC\Cplx

%%%%%%%%%%%%%%%%%%%%%%%%%%%%%%%%%%%%%%%%%%%%%%%%%%%%%%%%%%%%%%%%%%%%%%%%%%%%%%%%%%%%%%%
%%%%%%%%%%%%%%%%%%%%%%%%%%%%%%%%%%%%%%%%%%%%%%%%%%%%%%%%%%%%%%%%%%%%%%%%%%%%%%%%%%%%%%%
% 
% The main document start here.

% The following commands set up the material that appears in the header.
\doclabel{Math 401: Homework 1}
\docauthor{Your name goes here!}
\docdate{August 24, 2020}

\begin{document}

\begin{exercise}{1.3.9}  
\begin{enumerate}[(a)]
\item
If $\sup A < \sup B$ then show that there exists an element $b\in B$ that is an upper bound for $A$.
\item Give an example to show that this is not necessarily the case
if we we only assume $\sup A \le \sup B$.
\end{enumerate}
\end{exercise}
\begin{proof}[Proof (a)]
\end{proof}

Example for (b):


\begin{exercise}{1.3.11 }
Decide if the following statements are true.  Give a short proof
for the true statements and a counterexample for the false statements.
\begin{enumerate}[(a)]
\item If $A$ and $B$ are nonempty, bounded, and satisfy $A\subseteq B$
then $\sup A\le \sup B$.
\item If $\sup A< \inf B$ for sets $A$ and $B$, then there exists
$c\in\Reals$ such that $a<c<b$ for all $a\in A$ and $b\in B$.
\item If there exists $c\in\Reals$ satisfying $a<c<b$ for all
$a\in A$ and $b\in B$ then $\sup A< \inf B$. 
\end{enumerate}
\end{exercise}
\begin{proof}

\end{proof}



\begin{exercise}{First Edition 1.4.1}
Recall that $\mathbb{I}$ stands for the set of irrational numbers.
\begin{enumerate}[(a)]
\item Show that if $a,b\in\Rats$ then $ab$ and $a+b\in\Rats$ as well.
\item Show that if $a\in\Rats$ and $t\in\mathbb{I}$ then $a+t\in\mathbb{I}$ and if $a\neq 0$ then $at\in\mathbb{I}$ as well.
\item Part (a) says that the rational numbers are closed under multiplication
and addition.  What can be said about $st$ and $s+t$ when $s,t\in\mathbb{I}$?
\end{enumerate}
\end{exercise}
\begin{proof}
\end{proof}

\begin{exercise}{First Edition 1.4.2}
Let $A\subseteq \Reals$ be nonempty and bounded above. Let $s\in\Reals$ have
the property that for all $n\in\Nats$, $s+(1/n)$ is an upper bound for $A$
but $s-(1/n)$ is not an upper bound for $A$.  Show that $s=\sup A$.
\end{exercise}
\begin{proof}
\end{proof}

\begin{exercise}{First Edition 1.4.3} Show that $\cap_{n=1}^\infty (0,1/n)=\emptyset$.
\end{exercise}
\begin{proof}
\end{proof}

\begin{exercise}{First Edition 1.4.4}

Let $a<b$ be real numbers and let $T=[a,b]\cap\Rats$.
Show that $\sup T=b$.
\end{exercise}
\begin{proof}
\end{proof}

\begin{exercise}{First Edition 1.4.5} Use Exercise 1.4.1 to provide a proof of Corollary 1.4.4 (Density of Rational Numbers) by considering real numbers $a-\sqrt{2}$ and $b-\sqrt{2}$.
\end{exercise}
\begin{proof}
\end{proof}



\end{document}