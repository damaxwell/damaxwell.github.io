%%%%%%%%%%%%%%%%%%%%%%%%%%%%%%%%%%%%%%%%%%%%%%%%%%%%%%%%%%%%%%%%%%%%%%%%%%%%%%%%%%%%%%%
%%%%%%%%%%%%%%%%%%%%%%%%%%%%%%%%%%%%%%%%%%%%%%%%%%%%%%%%%%%%%%%%%%%%%%%%%%%%%%%%%%%%%%%
% 
% This top part of the document is called the 'preamble'.  Modify it with caution!
%
% The real document starts below where it says 'The main document starts here'.

\documentclass[12pt]{article}

\usepackage{amssymb,amsmath,amsthm}
\usepackage[top=1in, bottom=1in, left=1.25in, right=1.25in]{geometry}
\usepackage{fancyhdr}
\usepackage{enumerate}

% Comment the following line to use TeX's default font of Computer Modern.
\usepackage{times,txfonts}

\newtheoremstyle{homework}% name of the style to be used
  {18pt}% measure of space to leave above the theorem. E.g.: 3pt
  {12pt}% measure of space to leave below the theorem. E.g.: 3pt
  {}% name of font to use in the body of the theorem
  {}% measure of space to indent
  {\bfseries}% name of head font
  {:}% punctuation between head and body
  {2ex}% space after theorem head; " " = normal interword space
  {}% Manually specify head
\theoremstyle{homework} 

% Set up an Exercise environment and a Solution label.
\newtheorem*{exercisecore}{Exercise \@currentlabel}
\newenvironment{exercise}[1]
{\def\@currentlabel{#1}\exercisecore}
{\endexercisecore}

\newcommand{\localhead}[1]{\par\smallskip\noindent\textbf{#1}\nobreak\\}%
\newcommand\solution{\localhead{Solution:}}

%%%%%%%%%%%%%%%%%%%%%%%%%%%%%%%%%%%%%%%%%%%%%%%%%%%%%%%%%%%%%%%%%%%%%%%%
%
% Stuff for getting the name/document date/title across the header
\makeatletter
\RequirePackage{fancyhdr}
\pagestyle{fancy}
\fancyfoot[C]{\ifnum \value{page} > 1\relax\thepage\fi}
\fancyhead[L]{\ifx\@doclabel\@empty\else\@doclabel\fi}
\fancyhead[C]{\ifx\@docdate\@empty\else\@docdate\fi}
\fancyhead[R]{\ifx\@docauthor\@empty\else\@docauthor\fi}
\headheight 15pt

\def\doclabel#1{\gdef\@doclabel{#1}}
\doclabel{Use {\tt\textbackslash doclabel\{MY LABEL\}}.}
\def\docdate#1{\gdef\@docdate{#1}}
\docdate{Use {\tt\textbackslash docdate\{MY DATE\}}.}
\def\docauthor#1{\gdef\@docauthor{#1}}
\docauthor{Use {\tt\textbackslash docauthor\{MY NAME\}}.}
\makeatother

% Shortcuts for blackboard bold number sets (reals, integers, etc.)
\newcommand{\Reals}{\ensuremath{\mathbb R}}
\newcommand{\Nats}{\ensuremath{\mathbb N}}
\newcommand{\Ints}{\ensuremath{\mathbb Z}}
\newcommand{\Rats}{\ensuremath{\mathbb Q}}
\newcommand{\Cplx}{\ensuremath{\mathbb C}}
%% Some equivalents that some people may prefer.
\let\RR\Reals
\let\NN\Nats
\let\II\Ints
\let\CC\Cplx

%%%%%%%%%%%%%%%%%%%%%%%%%%%%%%%%%%%%%%%%%%%%%%%%%%%%%%%%%%%%%%%%%%%%%%%%%%%%%%%%%%%%%%%
%%%%%%%%%%%%%%%%%%%%%%%%%%%%%%%%%%%%%%%%%%%%%%%%%%%%%%%%%%%%%%%%%%%%%%%%%%%%%%%%%%%%%%%
% 
% The main document start here.

% The following commands set up the material that appears in the header.
\doclabel{Math 401: Homework 1}
\docauthor{Your name goes here!}
\docdate{August 24, 2020}

\begin{document}

\begin{exercise}{1.2.6 [Modified]}
Use the triangle inequality to establish the following inequalities:
\begin{enumerate}[(a)]
\item $|a-b| \le |a| + |b|$;
\item $||a|-|b|| \le |a-b|$.
\end{enumerate}
\end{exercise}
\begin{proof}
\end{proof}

\begin{exercise}{1.2.7(b), (d)}
Given a function $f$ and a subset $A$ of its domain, let $f(A)$ represent the range of $f$ over the set A;
that is, $f(a)=\{f(x):x\in A\}$.
\end{exercise}
\begin{enumerate}
\item[(b)] Find two sets $A$ and $B$ for which $f(A\cap B) \neq f(A)\cap f(B)$.
\item[(d)] Form and prove a conjecture concerning $f(A\cup B)$ and $f(A)\cup f(B)$.
\end{enumerate}
\begin{proof}[Proof (b)]
\end{proof}
\begin{proof}[Proof (d)]
\end{proof}

\begin{exercise}{1.2.11}
Form the logical negation of each claim. Do not use the easy way out: "It is not the case that$\ldots$" 
is not permitted
\begin{enumerate}[(a)]
\item For all real numbers satisfying $a<b$, there exists $n\in\Nats$ such that $a+(1/n)<b$.
\item There exist a real number $x>0$ such that $x<1/n$ for all $n\in\Nats$.
\item Between every two distinct real numbers there is a rational number.
\end{enumerate}
\end{exercise}
\solution
\begin{enumerate}[(a)]
\item 
\item 
\item 
\end{enumerate}

\begin{exercise}{[1.2 Supplement]} Show that the sequence $(x_1, x_2, x_3,\ldots)$ defined in Example
1.2.7 is bounded above by 2.  That is, show that for every $i\in\Nats$, $x_i\le 2$.
\end{exercise}
\begin{proof}
\end{proof}


\begin{exercise}{1.3.5}  Let $A$ be bounded above and let $c\in\Reals$.
Define  $cA = \{ca:a\in A\}$.
\begin{enumerate}[(a)]
\item If $c\ge 0$, show that $\sup(cA) = c\sup(A)$.
\item Postulate a similar statment for $\sup(cA)$ when $c<0$.
\end{enumerate}
\end{exercise}
\begin{proof}[Proof (a)]
\end{proof}

Statement for part (b): 


\begin{exercise}{1.3.7} Prove that if $a$ is an upper bound for $A$ and if $a$ is also an element of $A$,
then $a=\sup A$.
\end{exercise}
\begin{proof}
\end{proof}

\begin{exercise}{1.3.8} Compute, without proof, the suprema and infima of the 
following sets.
\begin{enumerate}[(a)]
\item $\{m/n: \text{ $m,n\in\Nats$ with $m<n$} \}$.
\item $\{(-1)^m/n: n,m\in\Nats\}$.
\item $\{n/(3n+1): n\in\Nats\}$.
\item $\{m/(m+n):m,n\in\Nats\}$.
\end{enumerate}
\end{exercise}
\solution
\begin{enumerate}[(a)]
\item
\item
\item
\item
\end{enumerate}

\end{document}