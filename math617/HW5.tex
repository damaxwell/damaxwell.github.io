\documentclass[minion]{homework}
\newcommand{\Reals}{\mathbb{R}}

\doclabel{Math F651: Homework 5}
%\docauthor{Your name here.}
\docdate{Due: February 28, 2018}
\begin{document}
\begin{aproblems}

Recall that $\hat L^1[a,b]$ is the set of equivalence classes
of Riemannian integrable functions on $[a,b]$, where $f\sim g$ if
\[
\int_a^b |f-g| = 0.
\]

\hproblem Show how to impose a suitable vector space structure
on $\hat L^1[a,b]$.  For the purposes of this exercise you must define
\[
[f]+[g]
\]
and
\[
\alpha[f]
\]
and show that this definition is well-defined.  Then, rather than showing
that all the vector space axioms hold, just show these two:
\begin{enumerate}
	\item There exists an element $Z\in \hat L^1[a,b]$ with $Z+F=F$ for
	all $F\in L^1[a,b]$.
	\item For all $\alpha\in\Reals$ and $F,G\in L^1[a,b]$, $\alpha(F+G)=\alpha F + \alpha G$.
\end{enumerate}

\hproblem Show that $\hat L^1[a,b]$
\[
||F|| = \int_a^b f
\]
where $f$ is any function with $[f]=F$ is a well-defined norm on $\hat L^1[a,b]$.

\hproblem Show that $\hat L^1[a,b]$ is not complete.  Here's one approach on $[0,1]$.  Consider the functions
\[
f_n(x) = \begin{cases} x^{-1/2} : x\ge 1/n\\
0 : \text{otherwise}.
\end{cases}
\]
Show that the sequence $[f_n]$ is Cauchy in $\hat L^1[a,b]$.  And then
show that if $f$ is a Rieman integrable function that $f_n\not\to f$.
Hint: if $f$ is Riemann integrable, there is a number $M$ with $|f|\le M$.
Now show that $||[f]-[f_n]||\ge 1/(2M)$ if $n$ is sufficiently large. 

\hproblem Using the rules presented in class for $L^1[a,b]$ show that if $f$ is
a representative of both $F$ and $G$ in $L^1[a,b]$ then $F=G$.

\hproblem TBA

\end{aproblems}
\end{document}

