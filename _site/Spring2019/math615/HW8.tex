\documentclass[minion]{homework}
\newcommand{\Reals}{\mathbb{R}}
\doclabel{Math F615: Homework 8}
\docdate{Due: March 29, 2019}
\usepackage{graphicx}
\usepackage{enumerate}
\newcommand{\bfx}{\mathbf{x}}
\newcommand{\bfv}{\mathbf{v}}

\begin{document}

\begin{problems}

\problem  Text, 4.20

\problem Text, 4.5

\problem  Consider the following method for solving the advection equation 
$u_t + au_x = 0$, where $a$ is constant:
\[
u_{i,j+1} = u_{i,j-1} + \frac{ak}{h}(u_{i-1,j}+u_{i+1,j})
\]
This is the leapfrog method.

\begin{subproblems}
\item Determine the order of accuracy of this method (in both space and time). The answer will be in the form $\tau(x,t)= O(k^p) +O(h^q)$; determine $p,q$.

\item This scheme can be derived by applying the method of lines.  
Do this as follows.
\begin{enumerate}[i)] 
	\item First, discretize in space using centered differences
to obtain a system of ODEs
\[
U'(t) = A U(t)
\]
for some matrix $A$.  What are the entries of the matrix $A$
assuming $x\in[0,1]$ is the space domain and periodic
boundary conditions are used ($u(0,t)=u(1,t)$?
\item Then use an ODE discretization method to derive the
whole scheme.  What rule from Table 1.1 on page 7 should be
applied at this step to derive the leapfrog scheme?
\end{enumerate}
\item What are the eigenvalues of the matrix $A$ from the previous derivation? 

\item Compute the region of absolute stability for the ODE discretization
method used in part b).  Then discuss the stability of the leap frog method
using this information and your computation from part c).

\item Implement this leapfrog method on the following periodic boundary condition problem: $x \in [0, 1], a = 0.5, T = 10, u(x, 0) = \sin(6\pi x)$. 
To make the implementation work you will have to compute the first step by some other scheme; describe and justify what you do.

\item What is the exact solution to the problem in part e)? Use
\[
h = 0.1, 0.05, 0.02, 0.01, 0.005, 0.002
\]
and $k = h$ and show a log-log convergence plot using the infinity norm for the error. What $O(h^p)$ do you expect for the rate of convergence, and what do you measure?
\end{subproblems}

\problem Turn in a solution to problem 4 from the midterm. Your solution for part b) must concretely discuss as part of the solution the eigenvalues of the matrix $D$ and the interaction of those eigenvalues with the region of absolute stability.  Hint: In theory, an $O(k^4)+O(h^2)$ method could use a time step of the size of $k=O(\sqrt{h})$ and have error that balances between time and space contributions.  But instead, this method will require $k=O(h^2)$.  Why?

\end{problems}

\end{document}