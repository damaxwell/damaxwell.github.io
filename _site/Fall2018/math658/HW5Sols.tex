\documentclass[minion]{homework}
\usepackage{graphicx}
\usepackage{cmacros}
\newcommand{\maple}[1]{{\tt\bf #1}}
\doclabel{Math F658: Homework 5 Solutions}
\docdate{October 16, 2018}

\begin{document}

\begin{aproblems}

\hproblem SR 6.3
\solution
The vector from event $A$ to $B$ has direction $U$ and
interval $\sigma$.  Since $U$ has interval $c$,
$$
B = A + \frac{\sigma}{c} U.
$$
Similarly, $C = B + \frac{\tau}{c}V$ and $C = A + \frac{\tau'}{c} V'$.
Thus
\[
A + \frac{\tau'}{c} V' = A + \frac{\sigma}{c} U + \frac{\tau}{c}V.
\]
Subtracting $A$ and multiplying by $c$ obtains the relation
\[
\tau' V' = \sigma U + \tau V.
\]
But then
\[
\tau'^2c^2 = g(\tau' V',\tau' V') = g(\sigma U + \tau V,\sigma U + \tau V) = \sigma^2 c^2 + \tau^2 c^2 + 2 \sigma \tau g(U,V).
\]
In the frame in which the traveler is at rest in the first part of his journey,
$U=(c,0)$ and $V = \gamma(v)(c,v)$ so $g(U,V)=c^2\gamma(v)$.  Thus
\[
\tau'^2 = sigma^2 + \tau^2 + 2 \sigma \tau \gamma(v).
\]
Since $v\neq 0$, $\gamma(v)>1$.  And since $\tau,\sigma>0$,
\[
\tau'^2 = sigma^2 + \tau^2 + 2 \sigma \tau \gamma(v) > sigma^2 + \tau^2 + 2 \sigma \tau = (\sigma+\tau)^2.
\]
We conclude that $\tau' > \sigma+\tau$.

Of course, in classical mechanics, the time difference between the two paths is identical, $\tau' = \sigma + \tau$.  

The interesting phenomenon here is that the \textbf{longest} path from $A$ to $C$
is the one for the non-accelerating traveler.
\hproblem SR 6.4
\solution
We may assume the traveler is traveling in the $t$, $x$ plane and we will ignore the other directions.  So $\alpha(\tau) = (ct(\tau),x(\tau))$ is its path
parameterized by proper time.  Now $\alpha'$ always has length $c$; this is
what it means to be parameterized by proper time.  So for each $\tau$ there
is a uniqe rapidity $\psi(\tau)$ such that the 4-velocity
\[
V = \alpha'(\tau) = c(\cosh(\psi),\sinh(\psi)).
\]
Taking another derivative with respect to $\tau$,
\[
A = \alpha''(\tau) = c(\sinh(\psi),\cosh(\psi))\frac{d\psi}{d\tau}
\]
Since $(\sinh(\psi),\cosh(\psi))$ is spacelike with interval $-1$,
\[
g(A,A) -c^2\left(\frac{d\psi}{d\tau}\right)^2
\]

\hproblem SR 6.5
\solution
Let 
\begin{align}
\alpha_A(\tau) &= (c^2/a) [ \sinh(a\tau/c), -\cosh(a\tau/c)] \\
\alpha_B(\tau) &= (c^2/a) [ \sinh(a\tau/c), \cosh(a\tau/c)]
\end{align}
be the paths of the two rockets that are accelerating in opposite directions
with acceleration $a$.  Then
\[
Z(\tau) = \alpha_B(\tau) - \alpha_A(-\tau) = \frac{2c^2}{a} [ \sinh(a\tau/c), \cosh(a\tau/c)].
\]
Thus $g(Z,Z)=-4c^4/a^2$ for all $\tau$.  Moreover,
\begin{align}
\alpha_A'(-\tau) &= c [ \cosh(-a\tau/c), -\sinh(-a\tau/c)]  = 
c [ \cosh(a\tau/c), \sinh(a\tau/c)]\\
\alpha_B'(\tau) &= c [ \cosh(a\tau/c), \sinh(a\tau/c)]
\end{align}
which are identical.  And
\[
g(\alpha_A'(-\tau),Z(\tau)) = \frac{2c^3}{a} \left[ \sinh(a\tau/c)\cosh(a\tau/c)
- \sinh(a\tau/c)\cosh(a\tau/c)\right] = 0.
\]
So $Z$ is a spacelike vector orthogonal to $\alpha_A'(-\tau)$ and 
therefore rocket A sees the event $\alpha_A(-\tau)$ as simultaneous
with the event $\alpha_B(\tau)$, at a constant distance $2c^2/a$.
The same holds for rocket $B$ since $\alpha_B'(\tau)=\alpha_A'(-\tau)$.

This is a difficult problem to reconcile with our intuition from classical mechanics.  The key observation is that as rocket $A$ travels into the future,
the parts of $B$'s worldline that $A$ deduces are 'now' actually travel back into the past.  So even though $A$ and $B$ are accelerating in opposite directions, in effect $A$ is seeing the course of time for $B$ run in reverse.  At each moment the velocities of $A$ and $B$ are deemed equal by $A$ and thus it is no surprise to find that the distance between the rockets remains the same.

Sketching the world lines of $A$ and $B$ it's easy to see, however, that 
light from rocket $A$ will never reach rocket $B$.  So although $B$ is a constant distance from $A$, according to $A$, the rocket $A$ never knows this.

\hproblem 
Let $\kappa(s)$ be a function on $\Reals$ and let
\begin{equation}
\phi(s) = \frac{1}{c} \int_0^s \kappa(r)\; dr.
\end{equation}
Show that
\begin{equation}
\alpha(s) = c \int_0^s (\cosh(\phi(s)),\sinh(\phi(s))\; ds
\end{equation}
is pararameterized by proper time and has a 4-acceleration with
size $|\kappa(s)|$.  What does the sign of $\kappa$ tell you?
\solution
From the Fundamental Theorem of Calculus,
\[
\phi'(s) = \frac{1}{c} \kappa(s)
\]
and
\[
\alpha'(s) = c (\cosh(\phi(s)),\sinh(\phi(s)))
\]
Evidently $g(\alpha',\alpha')=c^2$ and thus $\alpha$ is parameterized
by proper time.  Moreover,
\[
\alpha''(s) = c(\sinh(\phi(s)),\cosh(\phi(s)))\phi'(s) = 
(\sinh(\phi(s)),\cosh(\phi(s)))\kappa(s).
\]
Therefore $g(\alpha',\alpha') = - \kappa^2 $ and curve has acceleration 
of size $\kappa$.  Since the $x$-component of $\alpha''$ is 
$\cosh(\phi(s))\kappa(s)$, and since $\cosh(\phi)>0$, the sign of $\kappa$
determines if the acceleration is to the left or to the right.

\hproblem Using some kind of computer technology, generate a graph of a curve in spacetime with acceleration 
\begin{equation}
\kappa(s) = \sin(s)
\end{equation}
over the interval $s\in [0,2\pi]$.
\solution
We use units in which $c=1$.

Following the recipe from the previous problem, if 
$\kappa(s)=\sin(s)$ we can take
\[
\phi(s) = \frac{1}{c} \int_0^s \sin(r)\; dr =  1-\cos(s)
\]
But for purposes of our diagram, it's more convenient to take
$\phi(s)=-\cos(s)$, which does not change $\phi'(s)$ and simply corresponds
to an overall boost.  Then
\[
\alpha(s) = \int_0^s (\cosh(\cos(s)),-\sinh(\cos(s))\; ds
\]
This integration can be computed numerically, e.g. python's \texttt{scipy.integrate.quad}.

Here's my code
\begin{verbatim}
from math import *
from scipy.integrate import quad
import numpy as np
import matplotlib.pyplot as pp

def x(s):
	return quad( lambda z: sinh(-cos(z)),0,s)

def t(s):
	return quad(lambda z: cosh(-cos(z)),0,s)

tau = np.linspace(0,2*pi,100)

X = [x(s) for s in tau]
T = [t(s) for s in tau]

pp.plot(X,T,color='blue',linewidth=1.5)
tmax = np.max(T)
pp.plot([0,tmax],[0,tmax],color='green',linewidth=2)
pp.plot([0,-tmax],[0,tmax],color='green',linewidth=2)
pp.axis('scaled')
pp.xlabel('$x$')
pp.ylabel('$t$')
pp.savefig('HW5_f1.pdf')
pp.show()
\end{verbatim}

\includegraphics[width=5in]{HW5_f1.pdf}

\hproblem SR 7.1
\solution
Before the collision, we have momenta $P_1$ and $P_2$ with
$g(P_1,P_1)=M^2c^2$ and $g(P_2,P_2)= m^2 c^2$. After the collision
we have momentum $P=P_1+P_2$.  Moreover, $c^2g(P,P)$ is the square
of the rest energy of a particle with momentum $P$: after all,
in its rest frame $P=(cm_0,0)$ and $c^2 g(P,P)= c^2 c^2 m_0^2 = (m_0c^2)^2$
as desired.  Thus
\[
(E')^2 = c^2 g(P,P)= c^2g(P_1+P_2,P_2+P_2) = c^2 g(P_1,P_1) + c^2 g(P_2,P_2) + 2c^2 g(P_1,P_2).
\]
Now $g(P_1,P_1)=c^2M^2$ and $g(P_2,P_2)=c^2m^2$.  Moreover,
since $P_1$ has energy $E$ in the lab fame,
\[
P_1 = (E/c,*).
\]
Sicne $P_2=(cm,0)$ in the lab frame, $g(P_1,P_2) = (E/c)cm = Em$.
Thus
\[
(E')^2 = c^2( c^2M^2 + c^2m^2 + 2Em)
\]
as desired.

\end{aproblems}
\end{document}