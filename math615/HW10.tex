\documentclass[minion]{homework}
\newcommand{\Reals}{\mathbb{R}}
\doclabel{Math F615: Homework 10}
\docdate{Due: April 12, 2021}
\usepackage{graphicx}
\usepackage{enumerate}
\newcommand{\bfx}{\mathbf{x}}
\newcommand{\bfv}{\mathbf{v}}

\begin{document}

\begin{problems}

\problem  Text, 5.2

\problem  This variation of text, 5.3.  This problems concerns the wave equation $u_{tt}=c^2u_{xx}$ with initial data $u(x,0)$ and $u_t(x,0)$.
\begin{subproblems}
\item  Use the one-sided $O(h^2)$ approximation for the first
derivative from Table 1.1 on page 7 to derive an approximation for $u_t(x,0)$.
\item The approximation you just wrote down involves values of $u$ at timesteps $t_0$, $t_1$ and $t_2$.  The centered difference strategy for the wave equation also yields an equation involving values of $u$ at
timesteps $t_0$, $t_1$ and $t_2$.  Combine these two equations together 
to eliminate the values at $t_1$ to get an equation that relates values at $t_2$ to values at $t_0$.
\item Explain why the numerical scheme you just wrote down does not satisfy the CFL condition.
\item Instead, introduce an $O(h^2)$ approximation for the initial condition by introducing the (unknown) value of $u$ at $t_{-1}$.  This is a so-called ghost point.
\item Combine the equation you just derived together with the 
centered difference strategy for the wave equation to obtain a technique
for computing an approximation of $u$ at $t_1$ from initial conditions.  
\item Show that the equation you just derived satisfies the CFL condition.
\end{subproblems}

% condition.

%  Now apply both schemes to the wave equation $u_{tt}=u_{xx}$ for $0\le x\le 1$
% with initial conditions
% \begin{equation}
% \begin{aligned}
% u(x,0) &= \sin(\pi x)\\
% u_t(x,0) &= 0
% \end{aligned}
% \end{equation}

% Solve for $0\le t\le 2$ with $M=60, 100, 600, 1000, 6000$ time steps and with $N=(9/20) M$ space unknowns and generate convergence plots.  What orders
% of convergence do you observe?  Can you think of a reason to pick one of these
% methods over the other?

\problem  Text, 5.6

\end{problems}

\end{document}