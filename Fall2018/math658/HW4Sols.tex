\documentclass[minion]{homework}
\usepackage{cmacros}
\newcommand{\maple}[1]{{\tt\bf #1}}
\doclabel{Math F665: Homework 4 Solutions}
\docdate{Due: October 7, 2018}

\begin{document}

\begin{aproblems}

\hproblem SR 5.4

Let $O$ and $O'$ be two non-accelerating observers with inertial coordinate systems related
by a proper orthochronous Lorentz transformation $L$, with $E = L E'$. Show that
\begin{equation}
L = \begin{pmatrix} \gamma & -\gamma u_1'/c & -\gamma u_2'/c  & -\gamma u_3'/c\\
\gamma u_1/c &  * & * & *  \\
\gamma u_2/c &  * & * & *  \\
\gamma u_3/c &  * & * & *  \\
\end{pmatrix}
\end{equation}

\solution
The position of the origin in the $O'$ is given by $(t', 0, 0, 0)$.  Sending these events
through $L$ we find
\begin{equation}
\begin{pmatrix} ct & x & y & z \end{pmatrix} = ct' \begin{pmatrix} L^0_0 & L^1_0 & L^2_0 & L^3_0 \end{pmatrix}.
\end{equation}
Hence
\begin{equation}
t = L^0_0 t'
\end{equation}
and
\begin{equation}
u_1=\frac{dx}{dt} = \frac{1}{L^0_0} \frac{dx}{dt'} = c\frac{L^1_0}{L^0_0}.
\end{equation}
Thus 
\begin{equation}
L^{1}_0 = L^{0}_0 \frac{u_1}{c}.
\end{equation}
A similar equation holds for $y$ and $z$. Letting $\gamma$ denote the value of $L^0_0$
we find
\begin{equation}
L = \begin{pmatrix} \gamma & a_1 & a_2 & a_3 \\
\gamma u_1/c &  * & * & *  \\
\gamma u_2/c &  * & * & *  \\
\gamma u_3/c &  * & * & *  \\
\end{pmatrix}
\end{equation}
for some numbers $\gamma$, $a_1$, $a_2$, and $a_3$ .  But
\begin{equation}
L^t G L = G
\end{equation}
and consequently from the upper-left entry of this matrix equation
\begin{equation}
\gamma^2 (1- (u_1/c)^2 + (u_2/c)^2 + (u_3/c)^2) = 1.
\end{equation}
Since $L$ is orthochronus, $\gamma>0$ and hence
\begin{equation}
\gamma = \frac{1}{\sqrt{1- [(u_1)^2 + (u_2)^2 + (u_3)^2]/c^2}}=\gamma(u)
\end{equation}
where $u=\sqrt{u_1^2+u_2^2+u_3^2}$.

Similar considerations applied to $L^{-1}$ show that $L^{-1}$ has the form
\begin{equation}
L^{-1} = \begin{pmatrix} \gamma' & * & * & * \\
\gamma' u_1'/c &  * & * & *  \\
\gamma' u_2'/c &  * & * & *  \\
\gamma' u_3'/c &  * & * & *  \\
\end{pmatrix}
\end{equation}
where $\gamma'=\gamma(u')$ and where $u'=\sqrt{(u_1')^2+(u_2')^2+(u_3')^2}$.  But
\begin{equation}
L^t = G L^{-1} G = \begin{pmatrix} \gamma' & * & * & * \\
-\gamma' u_1'/c &  * & * & *  \\
-\gamma' u_2'/c &  * & * & *  \\
-\gamma' u_3'/c &  * & * & *  \\
\end{pmatrix}
\end{equation}
But
\begin{equation}
L^t = \begin{pmatrix} \gamma & \gamma u_1/c & \gamma u_2/c & \gamma u_3/c \\
a_1 &  * & * & *  \\
a_2 &  * & * & *  \\
a_3 &  * & * & *  \\
\end{pmatrix}
\end{equation}
from which we conclude that $\gamma'=\gamma$ and $a_i = - \gamma u_i'/c$
as required.

\hproblem SR 5.5
\solution

Let the matrics from the problem be $L_1$, $L_2$, $L_3$ and $L_4$
respectively.  It is an easy computation to show that
\[
L_1^t G L_1 = G
\]
and that $L_1$ has unit determinant.  Since it's upper-left entry is
positive, this is a proper, orthochronus Lorentz transformation.

It is a tedious computation (use Matlab/python/Julia/anything!) to check
that
\[
L_2^t G L_2 = G
\]
and that $\det(L_2)=-1$.  Since it's upper left entry is positive this
is an orthochronus but not proper Lorentz transformation.

Notice that $L_3$ is obtained from $L_2$ by multiplying the first column
by $-1$ and interchanging the middle columns.  Multiplying the first column
by $-1$ is effected by right multiplication by
\[
\begin{pmatrix} -1 & 0 \\
0 &I\end{pmatrix}
\]
which is evidently a Lorentz transformation. Interchanging the middle columns
is effected by right multiplication by
\[
\begin{pmatrix} 
1 &0& 0& 0\\
0 &0& 1& 0\\
0 &1& 0& 0\\
0 &0& 0& 1\end{pmatrix}
\]
which is readily seen to be a Lorentz transformation (it's a spatial reflection).  Hence the composition of $L_2$ with these two matrices
is also a Lorentz transformation.  It's determinant is the same as $L_2$
since the sign changes for the column interchange and the first column
multiplication cancel. Since the upper-left entry is negative, we find
$L_3$ is a non-proper, non-orthochronus Lorentz transformation.

By these same arguments, $L_4$ is a propoer, orthochronus Lorentz transformation.  Its determinant has the opposite sign of that of $L_3$,
and its upper-left entry is positive.

\hproblem SR 5.6

Let $V$ be a four-vector.  Show
\begin{itemize}
\item[i)] If $V$ is future pointing timelike, there is an inertial coordinate
system in which it has components $(a,0,0,0)$ with $a=\sqrt{g(V,V)}$.
\item[ii)] If $V$ is future pointing null, there is an inertial coordinate
system in which it has components $(1,1,0,0)$ with $a=\sqrt{g(V,V)}$.
\end{itemize}
\solution

For part i), assume that $V$ is future pointing and timelike. After applying
a spatial rotation we can assume that $V=(V^0,V^1,0,0)$. Pick $\phi$ by
\[
\tanh(\phi) = -\frac{V^1}{V^0},
\]
which is possible since $\tanh$ is invertible and $V^0>0$.
Setting $C=\cosh(\phi)$ and $S=\sinh(\phi)$, we define
\[
\begin{pmatrix} \hat V^0\\\hat V^1\end{pmatrix} \begin{pmatrix} C & S\\ S & C \end{pmatrix} \begin{pmatrix} V^0\\V^1\end{pmatrix} = \begin{pmatrix} C V^0 + S V^1\\ S V^0 + C V^1\end{pmatrix}
\]
But
\[
\hat V^1 S V^0 + C V^1 = \frac{C}{V^0} \left( -\tanh(\phi) + \frac{V^1}{V^0}\right).
\]
Since spatial rotations and the above boost are orhthochronus, so are their
composition and we find that we can take $V$ to a vector $\hat V =(a,0,0,0)$ by
an orthochronus Lorentz transformation.  Since $V$ is future pointing, so
is $\hat V$ and $a>0$. Moreover,
\begin{equation}
\sqrt{g(V,V)} = \sqrt{g(LV,LV)} =\sqrt{g(aT,aT)}=a\sqrt{g(T,T)} = a.
\end{equation}

 Part ii). Let $N$ be a null vector. Just as in
the our proof in class of part i), we can find an inertial frame where $N$ has
coordinates $N^1\ge 0$ and $N^2=N^3=0$.  Moreover, since $N$ is null,
\begin{equation}
(N^0)^2-(N^1)^2 = 0
\end{equation}
and since $N^0>0$ ($N$ is future pointing) and since $N^1>0$ we conclude
\begin{equation}
N = (a,a,0,0)
\end{equation}
for some $a>0$.

We now show by applying a boost that we can transform to $(1,1,0,0)$.
Note that for any rapidity $\phi$,
\begin{equation}
\begin{pmatrix} \cosh(\phi) & \sinh(\phi) \\ \sinh(\phi) & \cosh(\phi) \end{pmatrix}
\begin{pmatrix} a \\ a \end{pmatrix} = a\begin{pmatrix} e^\phi \\ e^\phi \end{pmatrix}.
\end{equation}
Letting $\phi=-\log(a)$ we find
\begin{equation}
a(e^\phi,e^\phi) = (1,1)
\end{equation}
as needed.  Hence we can boost in the $(t,x)$ plane by rapidity $\phi$ to transform
$N$ to $(1,1,0,0)$.

\hproblem SR 5.7

Show that
\begin{itemize}
\item[i)] The sum of future pointing timelike vectors is future pointing timelike.
\item[ii)] The sum of future pointing null vectors is future pointing timelike or
future pointing null, and is null if and only if the vectors are linearly dependent.
\item[iii)] Every four vector orthogonal to a timelike vector is spacelike.
\end{itemize}

\solution

\begin{lemma}\label{lem:mainlemma}
Let $X$ and $Y$ be causal and future pointing.  Then
\begin{equation}
g(X,Y) \ge ||X||||Y||
\end{equation}
with equality if and only if $X$ and $Y$ are linearly dependent.
\end{lemma}
\begin{proof}
First, suppose $X$ is timelike.  Without loss of generality, we may assume
\begin{equation}
X= (X^0,0,0,0)
\end{equation}
for some $X>0$.  Then
\begin{equation}
g(X,Y) = X^0 Y^0 = ||X|| Y^0
\end{equation}
Note that
\begin{equation}
||Y|| = \sqrt{(Y^0)^2-(Y^1)^2-(Y^3)^2-(Y^3)^2} \le |Y^0| = Y^0
\end{equation}
with equality if and only if $Y^1=Y^2=Y^3=0$ (i.e. if and only if $Y$
and $X$ are linearly dependent. Hence
\begin{equation}
g(X,Y) \ge ||X|| ||Y||
\end{equation}
with equality if and only if $X$ and $Y$ are linearly dependent.

The case where $Y$ is timelike follows from our previous argument and symmetry
of $g$, so we now suppose both $X$ and $Y$ are null. Since $||X||=||Y||=0$,
we need to show that
\begin{equation}
g(X,Y) \ge 0
\end{equation}
with equality if and only if $X$ and $Y$ are linearly dependent.
Without loss of 
generality we may assume 
\begin{equation}
X=(1,1,0,0)
\end{equation}
and hence
\begin{equation}
g(X,Y) = Y^0 - Y^1.
\end{equation}
Since $Y$ is null and future pointing
\begin{equation}
Y^0 = \sqrt{(Y^1)^2+(Y^2)^2+(Y^3)^2}.
\end{equation}
and hence
\begin{equation}
Y^0 \ge |Y^1|
\end{equation}
with equality if and only if $Y^2=Y^3=0$.
Now
\begin{equation}
Y^0 - Y^1 \ge |Y^1|-Y^1 \ge 0
\end{equation}
with equality if and only if $Y^2=Y^3=0$ and $Y^1\ge 0$.

Notice $Y=(Y^0,Y^1,0,)$ with $Y^1>0$ is equivalent to  $Y = Y^0(1,1,0,0)$
since $Y$ is null and future pointing.  Hence in the null case we find
\begin{equation}
g(X,Y)\ge 0
\end{equation}
with equality if and only if $Y=Y^0 X$ as needed.
\end{proof}

Now with the main solution.

For parts i) and ii), suppose $X$ and $Y$ are causal and future pointing.
Expressing the vectors with respect to an inertial frame, $(X+Y)^0=X^0+Y^0>0$
since $X^0$ and $Y^0$ are positive.  Hence $X+Y$ is future pointing.

Lemma \ref{lem:mainlemma} implies
\begin{equation}
\begin{aligned}
g(X+Y,X+Y) &= g(X,X) + g(Y,Y) + 2g(X,Y) \\
&\ge g(X,X) + g(Y,Y) + 2||X||||Y|| \\
&= (||X||+||Y||)^2 \\
&\ge 0
\end{aligned}
\end{equation}
with equality if and only if $X$ and $Y$ are linearly dependent and both null. Hence
$Z$ is timelike unless $X$ and $Y$ are null and linearly dependent, in which case $Z$ is null.

For part iii), suppose $X$ is timelike and $g(X,Y)=0$.  From our extended version
of part $i)$ we know that if $Y$ is causal and future pointing that $g(X,Y)>0$.
If $Y$ is causal and past pointing, then $-Y$ is causal and future pointing
and $g(X,-Y)>0$ and $g(X,Y)<0$.  Hence if $g(X,Y)=0$, then $Y$ is neither causal
future pointing, nor causal past pointing.  That leaves spacelike!



\hproblem SR 5.8
Let $X$ and $Y$ be future pointing and let $Z= X+Y$.  Then
\begin{equation}
\sqrt{g(Z,Z)} \ge \sqrt{g(X,X)} + \sqrt{g(Y,Y)}
\end{equation}
with equality if and only if $X$ and $Y$ are null and linearly dependent.

\solution
Lemma \ref{lem:mainlemma} implies
\begin{equation}
(||X||+||Y||)^2 = ||X||^2 + ||Y||^2 + 2||X||||Y|| \le ||X||^2 + ||Y||^2 + 2 g(X,Y) = g(X+Y,X+Y) = ||Z||^2
\end{equation}
with equality if and only if $X$ and $Y$ are are linearly dependent.
Hence
\begin{equation}
||Z|| \ge ||X||+||Y||
\end{equation}
with equality if and only if $X$ and $Y$ are linearly dependent.

This result is akin to the triangle inequality for Euclidean space:
\begin{equation}
||X+Y|| \le ||X|| + ||Y||
\end{equation}
with equality if and only if $X$ and $Y$ are linearly dependent.  The 
inequality in the Euclidean case points in the opposite direction.


\hproblem [The Peter Mulvey Observation] Your head will be younger than your feet by the time of your death.  Estimate, with justification, how much younger.

\solution
We apply equation (6.14) from the text,
\begin{equation}
\frac{d\tau}{dt} = \sqrt{1-R^2\omega^2/c^2}
\end{equation}
concerning the relationship between inertial time and proper time
for circular motion. The radii in question are the radius of the earth,
\begin{equation}
R_0=6400\mathrm{km}
\end{equation}
and radius, $R_1=R_0 + 2/1000$ corresponding to one's head.  We then have
\begin{equation}
\frac{d\tau_1}{dt}-\frac{d\tau_0}{dt} \approx -\frac{1}{1-R_0^2\omega^2/c^2} R_0(R_1-R_0)\frac{\omega^2}{c^2}=7.5\times 10^{-19}\;\mathrm{seconds.}
\end{equation}
Taking the elapsed $t$ to be 100 years we find
\begin{equation}
\tau_1-\tau_0 = 2.4\times 10^{-9},
\end{equation}
i.e about 3 nanoseconds. Of course, one is not standing one's whole life, and one is not always standing on the equator of the earth, so this estimate is an overestimate. Still, the net 
discrepancy on the order of one nanosecond, the about the time it takes my laptop to execute a single instruction.


\end{aproblems}
\end{document}