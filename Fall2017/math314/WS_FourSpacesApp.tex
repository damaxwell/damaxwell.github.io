\documentclass[minion]{homework}
\usepackage{cmacros}
\usepackage{graphicx}
\usepackage{color}
\usepackage[all,cmtip]{xy}
\def\vv{\mathbf{v}}
\def\vw{\mathbf{w}}
\def\vx{\mathbf{x}}
\def\vb{\mathbf{b}}
\def\vzero{\mathbf{0}}

% \newcommand\tomeW{{\color{red}\textbf{(W) (Hand this one in to David.)}}}
\newcommand\W{{\color{red}\textbf{(W) (Hand this one in to David.)}}}
\newcommand\tome{{\color{red}\textbf{(Hand this one in to David.)}}}
\doclabel{Math F314: Applications of the Fundamental Theorem of Linear Algebra}
\docdate{October 27, 2017}

\begin{document}

\begin{aproblems}
\vskip 0.5cm

The goal of this worksheet is for you to work through some applications
of the Fundamental Theorem of Linear Algebra:
\[
\begin{aligned}
C(A) &= N(A^T)^\perp\\
C(A^T) &= N(A)^\perp \\
\end{aligned}.
\]

For the first set of exercises consider
\[
A = \begin{bmatrix} 1 & 2 \\
-1 & -3 \\
3 & 0 \\
2 & 1
\end{bmatrix}.
\]
Since $A$ is tall and thin, we know that we cannot always solve
$A\mathbf x = \mathbf b$.  If we perform elimination, there will be 
at least two zero rows, and hence there will be at least two conditions
on $\mathbf b$ needed to ensure solvablility.  We would like to identify
these conditions.

\aproblem Consider the vector $\mathbf w = (-4, -3, 1, -1)$.  Show
that $A^T\mathbf{ w}=\mathbf{ 0}$.

\aproblem Use this vector to show that for $\mathbf{b}=(-1,1,1,1)$ there is
no solution of
\[
A\mathbf{x} = \mathbf{b}.
\]
Hint: Since $A^T\vw = \vzero$, $\vw^T A = \vzero^T $.  Now
try to combine this equation with $A\vx = \vb$.

\aproblem You just saw an example of a general principle.  If $A \mathbf x =  \mathbf b$ has a solution, then $\mathbf w \cdot \mathbf b$ for every
vector $\mathbf w$ such that $A^T\mathbf{w}=\mathbf 0$.  That is,
$C(A)\perp N(A^T)$.  We know something better: $C(A)= N(A^T)^\perp$.
So $A\mathbf x = \mathbf b$ will have a solution if and only if
$\mathbf b$ is in $C(A)$, which happens if and only if $\mathbf b$
is perpendicular to every vector in $N(A^T)$. And to detect
whether $\mathbf b$ is perpendicular to every vector in $N(A^T)$
you only need to show that $\mathbf b$ is perpendicular to every vector
in a basis for $N(A^T)$.  So now the search is on! We need to find a
basis for $N(A^T)$.  Start by performing elimination on $A^T$ to
arrive at row echelon form $U$.  Don't go all the way to $R$ yet!  At this point,
determine the dimension of $N(A^T)$.

\aproblem Now keep going to reduced row echelon form $R$. At this point,
find a basis for $N(A^T)$.

\aproblem For each vector $\mathbf w$ in your basis, double check 
and verify $A^T\mathbf w = \mathbf 0$.

\aproblem Is there a solution of $A\mathbf x = (1,-3,-9,-4)$?

\aproblem Is there a solution of $A\mathbf x = (-1,-3,-9,-4)$?
\vskip 10pt

\hrule 

Now suppose we have the vectors $\mathbf v_1 = (1,-1,1)$ and
$\mathbf v_2 = (1, 1, 2)$.  We would like to find 
a basis for the orthogonal complement of $V = \mathrm{span}(\vv_1,\vv_2)$.


\aproblem 
Before finding this orthogonal complement, let's get to know $V$
a bit better. First, show that $\vv_1$ and $\vv_2$ are linearly independent.

\aproblem What is the dimension of $V$? Why?

\aproblem What is the dimension of $V^\perp$? Why?

\aproblem To find a basis $V^\perp$, the strategy is to construct
a matrix $A$ such that $C(A)=V$.  Then use $V^\perp = C(A)^\perp = N(A^T)$.
So, write down a matrix $A$ such that $C(A)=V$.

\aproblem Now find a basis for $N(A^T)$, which is your desired basis for $V^\perp$.  Elimination strikes again!

\aproblem Does your basis have the expected number of elements? Compare with
your answer to problem 10.

\aproblem Verify each element of this basis is perpendicular
to both $\vv_1$ and $\vv_2$.

\aproblem I bet you have a different, favorite way of finding a vector
that is perpendicular to both $\vv_1$ and $\vv_2$ that you learned 
before you took linear algebra.  This technique is special to three dimensions, and only works there. Still, if you happen to be working
in three dimensions (like we are in this easy example), it's the way to go.
Go ahead and use this method to find a vector that
is perpendicular to both $\vv_1$ and $\vv_2$

\end{aproblems}
\end{document}
