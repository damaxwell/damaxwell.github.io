\documentclass[minion]{homework}
\usepackage{cmacros}
\usepackage{tensor}
\newcommand{\maple}[1]{{\tt\bf #1}}
\doclabel{Math F658: Homework 9}
\docdate{Due: November 5, 2018}
\def\Cplx{\mathbb{C}}
\begin{document}


\begin{aproblems}
\hproblem GR 1.1

\hproblem GR 1.2

\hproblem Suppose in $x$ coordinates a tensor $\tensor{T}{^a_b}$
has components equal to $\tensor{\delta}{^a_b}$.   Write down
in as simple a form you can the components in a different coordinate
system $\hat x$.  Then repeat this exercise if $T_{ab}=\delta_{ab}$
in $x$ coordinates.

\hproblem GR 4.4
Note: The notation $dt^2-dr^2-\sin^2(r)(d\theta^2)+\sin^2\theta d\phi^2$
with $x^0=1$, $x^1=r$, $x^2=\theta$ and $x^3=\phi$ is shorthand
for the metric $g_{ab}$ in these coordinates with
\[
[g_{ab}] = \begin{pmatrix} 
1 & 0 & 0 & 0 \\
0 & -1 & 0 & 0 \\
0 &  0 & -\sin^2r & 0 \\
0 & 0 & 0 & -\sin^2 r \sin^2 \theta \\
\end{pmatrix}
\]
Ignore the suggestion about Lagrange's equations.  Rather, compute the Christoffel symbols
and write down the geodesic equations directly.  

\end{aproblems}

\end{document}