\documentclass[minion]{homework}
\usepackage{cmacros}
\newcommand{\maple}[1]{{\tt\bf #1}}
\doclabel{Math F658: Homework 6}
\docdate{October 16, 2018}
\def\Cplx{\mathbb{C}}
\DeclareMathOperator{\Grad}{\rm Grad}
\DeclareMathOperator{\Div}{\rm Div}
\DeclareMathOperator{\grad}{\rm grad}
\begin{document}

\begin{aproblems}


\hproblem SR 7.2
\solution
From problem 7.1, we have
\[
(E')^2 = (m^2+m^2)c^4 + 2Emc^2
\]
where $E'$ is the rest energy of the $J/\psi$ particle and
$E$ is the energy of the positron in the lab frame.  But the 
rest energy of the $J/\psi$ particle is simply $Mc^2$ and hence
\[
M^2c^4 = 2m^2c^4 + 2E mc^2 
\]
and
\[
E = \frac{M^2c^2- 2m^2 c^2}{2m} = Mc^2\frac{M}{2m} - m^2c^2
\]
The rest energy of the positron in $mc^2$ and hence the excess is
\[
E - mc^2 = Mc^2\frac{M}{2m} - 2m^2c^2 = Mc^2 \left[\frac{M}{2m} - \frac{2m}{M} \right].
\]


Note that if instead we collide an electron and positron with opposite velocities $v$ and $-v$
the initial energy is
\begin{equation}
2mc^2\gamma(v)
\end{equation}
and the final energy is
\begin{equation}
Mc^2
\end{equation}
as the resulting particle is at rest.  Thus
\begin{equation}
2mc^2\gamma(v) = Mc^2
\end{equation}
The excess energy above the rest energy of the electron and the positron is then
\[
Mc^2 - 2mc^2 = Mc^2 \left[1-\frac{2m}{M}\right]
\]
This should be compared with
our previous excess
\[
Mc^2 \left[\frac{M}{2m} - \frac{2m}{M} \right]
\]
Since $m<<M$, we conclude $M/(2m) > 1$ and thus the execess energy for the equal
and opposite collision is less than that of the collision where the electron
is sationary.  The explanation for the difference is that in the first collision
there is additional energy due to the velocity of the $J/\psi$ particle.

\hproblem SR 7.3
\subsol
In the observer's inertial frame we have the momenta
\begin{equation}
\begin{aligned}
P_r&=m\gamma(u)(c,u)\\
P_e&=M\gamma(v)(c,-v)\\
P'_r &= m'\gamma(u')(c,u').
\end{aligned}
\end{equation}
Here the subscript $r$ denotes the rocket, the subscript $e$ denotes
the ejecta, and the prime denotes post-ejection.  From
conservation of 4 momentum we have
\begin{equation}\label{eq:Encons}
m\gamma(u) = M\gamma(v) + m'\gamma(u')
\end{equation}
and
\begin{equation}\label{eq:mcons}
m\gamma(u)u = -M\gamma(v)v + m'\gamma(u')u'.
\end{equation}
Using equation \eqref{eq:Encons} to replace $m\gamma(u)$ in equation
\eqref{eq:mcons} we conclude
\begin{equation}\label{eq:a}
M\gamma(v)(u+v) + m'\gamma(u')(u-u') = 0.
\end{equation}
Note also that since $P_r-P_e=P_r'$,
\begin{equation}
g(P_r-P_e,P_r-P_e) = g(P_r',P_r') = c^2(m')^2.
\end{equation}
But we can directly compute
\[
g(P_r-P_e,P_r-P_e) = g(P_r,P_r) +g(P_e,P_e) - 2g(P_r,P_e) = c^2m^2 + c^2M^2 -2g(P_r,P_e).
\]
Working in the rest frame of the rocket we see that 
$g(P_r,P_e) = \gamma(w)c^2mM$. Hence
\begin{equation}\label{eq:c}
m^2+M^2-2\gamma(w)mM = m'^2.
\end{equation}

\subsol
The equation
\begin{equation}\label{eq:trans}
\gamma(v)\begin{pmatrix} c \\-v\end{pmatrix} = \gamma(u)\gamma(w)\begin{pmatrix} 1 & u/c \\ u/c & 1\end{pmatrix}\begin{pmatrix} c \\-w\end{pmatrix}
\end{equation}
follows from the following facts:
\begin{itemize}
\item The rocket is traveling at velocity $u$ relative to the observer so 
the transformation from the rocket's frame to the observer's frame is
\begin{equation}
\gamma(u)\begin{pmatrix} 1 & u/c \\ u/c & 1\end{pmatrix}.
\end{equation}
\item The 4 velocity of the ejecta in the rocket's frame is $\gamma(w)(c,-w)$.
\item The 4 velocity of the ejecta in the observer's frame is $\gamma(v)(c,-v)$.
\end{itemize}
\subsol
Equation \eqref{eq:trans} can be equivalently written
\begin{equation}
\gamma(v)\gamma(u)\begin{pmatrix} 1 & -u/c \\ -u/c & 1\end{pmatrix}\begin{pmatrix} c \\-v\end{pmatrix} = \gamma(w)\begin{pmatrix} c \\-w\end{pmatrix}.
\end{equation}
Looking at the spatial part of this equation we conclude that
\begin{equation}\label{eq:b}
\gamma(v)\gamma(u)(u+v) = \gamma(w)w.
\end{equation}
From equations \eqref{eq:a} and  \eqref{eq:b}
\begin{equation}
mm'\gamma(u')\gamma(u)(u'-u) = mM\gamma(u)\gamma(v)(u+v) = mM\gamma(w)w.
\end{equation}
Equation \eqref{eq:c} then implies
\begin{equation}
mm'\gamma(u')\gamma(u)(u'-u) = \frac{1}{2}(m^2+M^2-(m')^2)w.
\end{equation}

\subsol
Setting $u'-u=\delta u$ and $m'-m=\delta m$ we find
\begin{equation}
mm'\gamma(u')\gamma(u) \delta u = \frac{1}{2}( -(m+m')\delta m +M^2)w
\end{equation}
and hence
\begin{equation}
mm'\gamma(u')\gamma(u) \frac{\delta u}{\delta m} = -mw-\frac{1}{2}\delta mw + \frac{1}{2}\frac{M^2}{\delta m}w.
\end{equation}
Using the fact that $m'\ra m$ and $u'\ra u$ as $\delta u,\delta m\ra 0$ we find
\begin{equation}
m^2\gamma(u)^2 \frac{d u}{d m} =- mw
\end{equation}
so long as $M^2/(\delta m)\ra 0$.  But equation \eqref{eq:c} ensures that $\delta m$
and $M$ are linearly related in the limit as $M\ra 0$, so indeed $M^2/(\delta m)\ra 0$.

\subsol
See text.


\hproblem SR 3.3

\hproblem SR 5.9
We note that
\[
\sigma^2 = c^2t^2 - \sum (x^i)^2
\]
and hence
\[
\Grad \sigma^2 = 2(ct,-(-x^1),-(-x^2),-(-x^3)) = 2X.
\]
Moreover, by the chain rule, for any functions $f:\Reals\ra\Reals$ and 
$g:\Reals^{1,3}\ra\Reals$,
\[
\mathrm{Grad}\; f\circ g(x) = f'(g(x))\mathrm{Grad}\; g.
\]
Hence
\[
\mathrm{Grad}\; f(\sigma^2) = 2f'(\sigma^2) X.
\]
Finally, we observe that
\[
\Div X = \partial_0 x^0 + \partial_1 x^1 + \cdots + \partial_3 x^3 = 4.
\]

Now consider $f(x)=x^{-1}$ so 
\[
f(\sigma^2) = \frac{1}{g(X,X)}
\]
Then 
\[
\mathrm{Grad} \frac{1}{g(X,X)} = -2\frac{1}{g(X,X)^2}X.
\]
Now for any function $h(x)$,
\[
\Div (h(x) X) = g(\Grad f,X) + f(x) \Div X
\]
and hence
\[
\Div \Grad \frac{1}{g(X,X)} = 4\frac{1}{g(X,X)^3}g(X,X) - \frac{1}{g(X,X)^2} \Div X =  4\left[\frac{1}{g(X,X)^2}-\frac{1}{g(X,X)^2}\right]=0.
\]
That is,
\[
\frac{1}{ct^2-(x^1)^2-(x^2)^2-(x^3)^2}
\]
solves the wave equation (off of the light cone).


\hproblem   
\begin{subproblems}
\item Let $\xi\in\Reals^3$, let $z\in\Cplx$ and let $u(t,x)=ze^{i(\xi\cdot x - c|\xi|t) )}$.
Show that $u$ is a complex valued solution of the wave equation.  Describe its
real part as a wave.  What is the speed of the wave? What direction is it travelling in? What is
its frequency?
\item Let $\hat f:\Reals^3\ra \Cplx$ be smooth and compactly supported (i.e., $f(\xi)=0$ for $\xi$
outside of some large ball). Show that
$$
U(t,x)=\int_{\Reals^3} \hat f(\xi) e^{i(\xi\cdot x - c|\xi|t) )} d\xi
$$
is a complex-valued solution of the wave equation (and hence its real and imaginary parts
both solve the wave equation).  How is the solution $U$ related to the kinds of solutions
described in part a)?
\item
Show that
\begin{equation}
u(t,x) = U(t,x) + U(-t,x)
\end{equation}
solves the wave equation with $u_t(0,x)=0$.
\item A result from Fourier analysis says that if $f:\Reals^3\ra \Cplx$
is, say, continuous and
\begin{equation}
\int_{\Reals^3} |f|^2
\end{equation}
is finite, then
\begin{equation}
f(x) =  \int_{\Reals^3} \hat f(\xi) e^{i\xi\cdot x } d\xi
\end{equation}
where
\begin{equation}
\hat f(\xi) = \frac{1}{(2\pi)^3} \int_{\Reals^3} f(x) e^{-i\xi\cdot x } dx.
\end{equation}
With this result in hand, describe a strategy for solving the initial value problem
\begin{align*}
u_{tt}-c^2\Delta u &= 0\\
u(0,x) &= \phi(x)\\
u_t(0,x) &= 0.
\end{align*}
\item Challenge: describe a strategy for solving the initial value problem
\begin{align*}
u_{tt}-c^2\Delta u &= 0\\
u(0,x) &= 0\\
u_t(0,x) &= \psi.
\end{align*}
What new issues appear compared to part \emph{d}?
\end{subproblems}
\solution
\subsol
We can write $z=re^{i\theta}$ for some $r>0$ and $\theta\in\Reals$.  Then
\[
u(t,x) = r e^{i(\xi \cdot x - c|\xi|t + \theta)}.
\]
Note that
\[
u_tt = -c^2|\xi|^2 u(t,x)
\]
and
\[
(\partial_i)^2 u = -(\xi^i)^2 u(t,x)
\]
for $i=1,2,3$.
Hence
\[
\Box u = \frac{1}{c^2} u_tt - \Delta u = (|\xi|^2-|\xi|^2) u = 0.
\]
So $u$ is a complex valued solution of the wave equation.

Its real part is
\[
\Re u = r\cos(\xi\cdot x - c |\xi| t + \theta )
\]
Let 
\[
f(s) = r\cos(|\xi| s + \theta).
\]
Then
\[
\Re u(t,x) = f(\mathrm{e}\cdot x - ct)
\]
where $\mathrm{e}=\xi/|\xi|.$  This exhibits $\Re u$ as a wave traveling in the
direction $\mathrm{e}$ with velocity $c$. Regarding its frequency, consider a sationary
observer starting at $x=0$. This observer sees
the function values
$$
\Re u(t,0) = f(-ct) = r\cos(|\xi|ct + \theta).
$$
Hence, in unit time, the argument to $\cos$ passes through $|\xi|c/(2\pi)$ periods.
The frequency is therefore $|\xi|c/(2\pi)$.

\subsol

Let
\[
U(t,x)=\int_{\Reals^3} \hat f(\xi) e^{i(\xi\cdot x - c|\xi|t) )} d\xi
\]
That $U(t,x)$ solves the wave equation comes from linearity of the wave equation,
together with the fact that we can commute integration and differentiation in this case.
One always has to worry about switching limiting operations, and the hypotheses
that $\hat f$ is smooth and compactly supported are sufficient (and are overkill).
Note that $U$ is a superposition of waves with frequencies $c|\xi|/(2\pi)$ travelling
in the directions $\xi/|\xi|$.  The coefficeint $\hat f(\xi)$ describes the amplitude
of the component wave with frequency $c|\xi|/(2\pi)$ and direction $\xi/|\xi|$.

\subsol

Note that $(\partial_t)^2 U(-t,x) = U_{tt}(-t,x)$ and hence $\Box (U(-t,x)) = (\Box U)(-t,x) = 0$.

That $u(t,x)=U(t,x)+U(-t,x)$ solves the wave equation follows from linearity.  Moreover
\[
u_t(0,x) = U_t(0,x) - U_t(0,x) = 0.
\]

That $\tilde u(t,x)=U(t,x)-U(-t,x)$ solves the wave equation also follows from linearity.  Moreover
\[
\tilde u(0,x) = U_t(0,x) - U_t(0,x) = 0.
\]

\subsol

Consider a solution of the form
\[
u(t,x) = U(t,x)+U(-t,x)
\]
where
\[
U(t,x) = \int \hat f(\xi) e^{i(\xi\cdot x - c|\xi|t)} \; d\xi.
\]
Note that
\[
u(0,x) = \int 2 \hat f(\xi) e^{i(\xi\cdot x} \; d\xi.
\]
But we can write
\[
\phi(x) = \int \hat \phi (\xi) e^{i(\xi\cdot x} \; d\xi
\]
where
\[
\hat \phi(\xi) = \frac{1}{(2\pi)^3} \int_{\Reals^3} \phi(x) e^{-i\xi\cdot x } dx.
\]
So we set $\hat f(\xi)=\hat \phi(\xi)/2$. Strictly speaking, this is a strategy, not
a solution.  In part b, in order to allow us to glibly interchange integration and differentiation,
I added the hypothesis that $\hat f$ was smooth and compactly supported.  
This need not be true for $\phi$.
Nevertheless, when $\phi$ is square integrable, it turns out that this strategy can be made
rigorous.

\subsol

Now consider a solution of the form
\[
u(t,x) = U(t,x)+U(-t,x)
\]
where
\[
U(t,x) = \int \hat f(\xi) e^{i(\xi\cdot x - c|\xi|t)} \; d\xi.
\]
Note that
\[
u_t(0,x) = \int -2c|\xi|i \hat f(\xi) e^{i(\xi\cdot x} \; d\xi,
\]
at least when $\hat f$  is compactly supported.  
We would like
\[
u_t(0,x) = \psi(x) = \int \hat \psi (\xi) e^{i(\xi\cdot x} \; d\xi
\]
where
\[
\hat \psi(\xi) = \frac{1}{(2\pi)^3} \int_{\Reals^3} \psi(x) e^{-i\xi\cdot x } dx.
\]
So we require
\[
\hat f(\xi) = -\frac{1}{2c|\xi|} \hat g(\xi).
\]
Again, we have all the caveats as before: $\hat f$ need not be smooth, nor need it be compactly
supported.  The division by $|\xi|$ is a new, troublesome wrinkle, and to make this a formal 
solution, one would need to also ensure that the singularity in $\hat g$ at $0$ does not
pose a problem.

\end{aproblems}
\end{document}