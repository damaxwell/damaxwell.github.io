\documentclass[minion]{homework}
\usepackage{cmacros}
\newcommand{\maple}[1]{{\tt\bf #1}}
\doclabel{Math F665: Homework 3 Solutions}
\docdate{Due: October 7, 2018}
\def\Int{\mathrm{Int}}
\usepackage{tensor}

\begin{document}

\begin{aproblems}

\hproblem SR 4.3

See text, page 176.

\hproblem SR 4.4

Two light sources are at rest distance $D$ apart 
and emit photons in the positive
$x$ direction.  Show that in a frame where the sources have velocity
$u$ along the $x$-axis, the photons have distance
\[
D\sqrt\frac{c-u}{c+u}
\]
apart.

\solution
Let the coordinates of the moving frame be denoted by $(\hat t,\hat x)$
so we can take
\[
\begin{pmatrix} c\hat t \\ \hat x\end{pmatrix} = \gamma \begin{pmatrix} 1 & u/c\\
u/c & 1\end{pmatrix} \begin{pmatrix} c t \\ x\end{pmatrix}
\]
Let $L$ denote this boost.

We will assume that the emissions occur at $t=0$ and
that source $S_1$ is at $x=0$ and source $S_2$ is at $x=D$.
In the frame where the sources are moving the emission
from $S^1$ is at $(0,0)$ again but the emission from $S^2$
has coordinates
\[
L \begin{pmatrix}0\\ D\end{pmatrix} 
=\gamma D\begin{pmatrix} u/c\\ 1\end{pmatrix}.
\]
Note that the emission from $S_2$ occured after the emission
at $S_1$ in the moving frame at time $c\hat t = \gamma D  (u/c)$.
At this time, the photon from $S_1$ will be at position
$c\hat t = \gamma D  (u/c)$ and  hence
\[
\Delta \hat x = \gamma D - \gamma D  (u/c) = 
D \frac{1- (u/c)}{\sqrt{1-(u/c)^2}} = D \sqrt{\frac{c-u}{c+u}}
\]
This distance remains constant since the photons are both travelling
to the right with speed $c$.

\hproblem The interval between events $E_1=(t_1,x_1)$ and $E_2=(t_2,x_2)$ in some inertial coordinate
system is
\begin{equation}
c^2(t_1-t_2)^2 - (x_1-x_2)^2.
\end{equation}
Suppose $\iota:\Reals^2\ra\Reals^2$ is a transformation that preserves the interval between any two
events.  Assuming that $\iota$ is affine, show that there is a (possibly non-proper or non-orthochronus) Lorentz transformation $L$ and a vector $b\in \Reals^2$ such that
\begin{equation}
\iota(E) = {\mathcal C}^{-1} L {\mathcal C} E + b.
\end{equation}
Here $\mathcal C$ is the $2\times 2$ diagonal matrix with diagonal entries $c$ and $1$.
Hint: This problem should feel very familiar! And take advantage of problem 4.2!

\solution
Suppose $\iota$ is affine and preserves the interval. Suppose first that $\iota$ takes the origin
to the origin, so $\iota$ is linear. So we can write
\begin{equation}
\iota( E) = C^{-1} L C E
\end{equation}
for some matrix $L$.

Let $E$ be an event.  Then
\begin{equation}
\Int(0,E) = E^t C^t G C E
\end{equation}
and
\begin{equation}
\Int(\iota(0),\iota(E)) = \Int(0,\iota(E)) =  \Int(0,C^{-1} L C E) = E^t C^t L^t G  L C E.
\end{equation}
So for all events $E$,
\begin{equation}
E^t C^t G C E = E^t C^t L^t G  L C E.
\end{equation}
Let $B$ be the symmetric bilinear form
\begin{equation}
B(E,F) = E^t C^t G  C F
\end{equation}
and let $\hat B$ be the symmetric bilinear form
\begin{equation}
\hat B(E,F) = E^t C^t L^t G  L C F.
\end{equation}
Since $B$ and $\hat B$ agree on the diagonal they are the same, and we conclude that
\begin{equation}
C^t G  C = C^t L^t G  L C.
\end{equation}
Multiplying on the right by $C^{-1}$ and on the left by $(C^t)^{-1} = C^{-1}$ we conclude that
\begin{equation}
G = L^t G  L.
\end{equation}

Since $G\indices{^0_0}=1$ we conclude that
\begin{equation}
(L\indices{^0_0})^2 - (L\indices{^0_1})^2-(L\indices{^0_2})^2-(L\indices{^0_3})^2=1
\end{equation}
and consequently $L\indices{^0_0}\neq 0$.  Moreover, since $\det G\neq 0$ we know that $\det L\neq 0$.
Thus there are four possibilities for the sign of $L\indices{^0_0}$ and $\det L$ (e.g., both positive, both negative, etc.).  If $L\indices{^0_0}>0$ and $\det L>0$ then we know from problem
4.2 that $L$ is a proper, orthochronus Lorentz transformation. On the other hand, let 
\begin{equation}
F=\begin{pmatrix} -1 & 0\\ 0 & 1 \end{pmatrix}\qquad R=\begin{pmatrix} 1 & 0\\ 0 & -1 \end{pmatrix}\qquad 
\end{equation}
It is easy to see that $F^t G F = G$ and hence
\begin{equation}
(FL)^t G FL = L^t F^t G F L = L^t G L = G.
\end{equation}
Similarly $(RL)^t G RL=G$ and $(RFL)^t G (RFL)=G$.  It is easy to see that one of
$L$, $RL$, $FL$, and $RFL$ has positive determinant and a positive upper-left entry.
Hence one of these is a proper orthochronus Lorentz transformation and $L$ is therefore
the composition of a proper orthochronus Lorentz transformation with a space or time reflection (or both).



\hproblem For simplicity the following problem is to be done in one space dimension.
Suppose in the frame of some inertial observer a function has the form
\begin{equation}
f(t,x) = \sin(\omega t)
\end{equation}
for some angular frequency $\omega$.  Now consider the frame of some observer traveling
with velocity $v$ relative to the original frame.  Determine the time difference between peaks
of the function as seen by the boosted observer.
\solution
Let $(\hat t, \hat x)$ be inertial coordinates for the boosted observer.  Then
\begin{equation}
\begin{pmatrix} ct \\ x \end{pmatrix} = \gamma(v) \begin{pmatrix} 1 & \frac{v}{c} \\ \frac{v}{c} & 1\end{pmatrix} \begin{pmatrix} c\hat t \\ \hat x \end{pmatrix}.
\end{equation}
In particular,
\begin{equation}
t = \gamma(v) \left[ \hat t + (v/c^2) \hat x\right].
\end{equation}
Thus the function for the boosted observer is
\begin{equation}
f(\hat t, \hat x) = \sin(\omega \gamma(v) ( \hat t + (v/c^2) \hat x)).
\end{equation}
Our observer has coordinates with $\hat x = 0$ and thus at time $\hat t$, the
observer measures the function values to be
\begin{equation}
\sin(\omega \gamma(v)\hat t).
\end{equation}
The observed period is therefore
\begin{equation}
\frac{2\pi}{\omega\gamma(v)}.
\end{equation}


\hproblem Pions are subatomic particles with a half life of $\Delta t=1.8\times10^{-8}$ seconds.  As a consequence, given a collection of pions left alone for a time $\Delta t$, half
of the pions will decay into other particles.

A beam of pions is traveling at a speed $v=0.99c$. Notice that in time $\Delta t$ the beam
travels
\begin{equation}
\Delta x = 0.99 c \Delta t = 5.35 \mathrm{m}
\end{equation}
and one might expect that the beam diminishes in intensity by one half every 5.35m.  Instead,
it deminishes by one half every 38m or so.  Explain the discrepancy.
\solution
Let $(\hat t, \hat x)$ be coordinates in the rest frame of the pions, and let $(t,x)$ 
be coordinates in the rest frame of the lab.  So
\begin{equation}
\begin{pmatrix} ct \\ x \end{pmatrix} = \gamma(v) \begin{pmatrix} 1 & \frac{v}{c} \\ \frac{v}{c} & 1\end{pmatrix} \begin{pmatrix} c\hat t \\ \hat x \end{pmatrix}.
\end{equation}
where $v/c=0.99$.  In the rest frame of the pions, the beam follows the worldline between $(0,0)$
and $(\Delta \hat t, 0)$ and diminishes intensity by one half.  Here $\Delta\hat t = 1.8\times 10^{-8}$ seconds.

Transforming to the lab frame, the beam starts at $(0,0)$ and reaches
\begin{equation}
\begin{pmatrix} ct \\ x \end{pmatrix} = \gamma(v) \begin{pmatrix} 1 & \frac{v}{c} \\ \frac{v}{c} & 1\end{pmatrix} \begin{pmatrix} c\Delta \hat t \\ 0 \end{pmatrix} = \gamma(v) \Delta \hat t
\begin{pmatrix} c \\ v \end{pmatrix}
\end{equation}
when the beam has diminished in intensity by one half.  Note that the $x$ coordinate is
\begin{equation}
\gamma(v) \Delta\hat t v = (7.08) \cdot 1.8\times 10^{-8} \cdot (.99c)  = 38m
\end{equation}

\end{aproblems}


\end{document}