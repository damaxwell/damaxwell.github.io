\documentclass[minion]{homework}
\usepackage{cmacros}
\newcommand{\maple}[1]{{\tt\bf #1}}
\doclabel{Math F658: Homework 6}
\docdate{Due: October 22, 2018}
\def\Cplx{\mathbb{C}}
\begin{document}

\begin{aproblems}

\hproblem Suppose a particle travels with a force law of the form
\[
\frac{dP}{d\tau} = A V
\]
where $P$ is its 4-momentum, $V$ is its 4-velocity and $A$ is a spacetime dependent matrix.  Assuming that $g(P,P)$ does not depend on $\tau$, show that
\[
A=GB
\]
where $B$ is antisymmetric.

\hproblem Let $\omega$ be a covector expressed in an inertial
coordinate system.  We define
\begin{equation}
(d\omega)_{ij} = \partial_i \omega_j - \partial_j \omega_i.
\end{equation}

\begin{subproblems}
\item
Show that in a second inertial coordinate system $\hat x = L x + Y$,
\begin{equation}
L^t \widehat{d\omega} L = d\omega.
\end{equation}
Here,
\begin{equation}
(\widehat{d\omega})_{ij} = \hat \partial_i \hat \omega_j - \hat \partial_j \hat \omega_i.
\end{equation}

\item If $f$ is a function, show $d(df)=0$.
\item Let $K=(K_1,K_2,K_3)$ and $L=(L_1,L_2,L_3)$ be triples of real numbers.  We define
the matrix
\begin{equation}
\mathcal{F}(K,L)= \begin{pmatrix} 0 & K_1 & K_2 & K_3 \\
-K_1& 0& L_1& -L_2 \\
-K_2& -L_1& 0& L_3 \\
-K_3& L_2& -L_2& 0\end{pmatrix}
\end{equation}
If $\omega = ( \omega_0, \overrightarrow{\omega})$, show that 
\begin{equation}
d\omega = \mathcal{F}(\partial_0 \overrightarrow{\omega}-\nabla \omega^0,\nabla \times 
\overrightarrow{\omega})
\end{equation}
Here we interpret $\overrightarrow{\omega}$ as the coordinates of a vector in $\Reals^3$
and $\nabla$ and $\nabla\times$ are the standard gradient and curl operators in $\Reals^3$.
\end{subproblems}

\hproblem Recall that
\begin{align}
*\mathcal{F}(R,S) &= \mathcal{F}(S,-R)\\
*d \mathcal{F}(R,S) &= [ \mathrm{div}\; S, -\nabla \times R + \partial_0 S].
\end{align}

\begin{subproblems}
\item Let $\omega$ be a 1-form.  Show $*dd\omega=0$.  (In fact, this shows $d^2=0
$ acting on $\Lambda^1$)
\item Let $F$ be a 2-form.  Show $\delta\delta F=0$.
\item Unwind the definitions and show that if $\omega$ is a 1-form, then
\begin{equation}
-\delta d \omega = \Box \omega - d\delta \omega.
\end{equation}
\end{subproblems}

\hproblem
Given a one-form $\omega$ we define the associated electric and magnetic
fields $E$ and $B$ by
\[
d\omega = \mathcal{F}(E,-cB).
\]
Recalling that Maxwell's equations are
\[
-\delta d\omega = \frac{1}{c\epsilon_0}(c\rho,-j)
\]
\
show that $E$ and $B$ satisfy
\begin{align}
\nabla\cdot E &= \frac{1}{\epsilon_0} \rho\\
\frac{1}{c}\partial_0 E + \nabla\times B &= \frac{1}{c^2\epsilon_0} j.
\end{align}
These are Gauss' Law and Ampere's equation respectively.

Then, from the fact that $\delta^2=0$ show that
\begin{align}
\nabla \cdot B &= 0\\
c\partial_0 B + \nabla\times E &= 0.
\end{align}
These are Gauss' Law for magnetism and Faraday's Law, respectively.

\hproblem  The fact that $d^2=0$ when acting on $\Lambda^0$ and $\Lambda^1$ 
has a partial converse.
Use the results on page 184 of the text to prove the following.
\begin{subproblems}
\item Suppose on some ball that $d\omega=0$ some some one-form $\omega$.  Show that there is a function $f$ on the ball
such that $\omega=df$.
\item Suppose on some ball that $*dF=0$ for some two-form $F$; this is equivalent to $dF=0$
for the map $d:\Lambda^2\ra\Lambda^3$ that we did not discuss in detail.  Show that there is a 
one-form $\omega$ with $F=d\omega$.
\end{subproblems}


\end{aproblems}
\end{document}