\documentclass[minion]{homework}
\usepackage{cmacros}
\usepackage{graphicx}
\usepackage{color}
\usepackage[all,cmtip]{xy}

% \newcommand\tomeW{{\color{red}\textbf{(W) (Hand this one in to David.)}}}
\newcommand\W{{\color{red}\textbf{(W) (Hand this one in to David.)}}}
\newcommand\tome{{\color{red}\textbf{(Hand this one in to David.)}}}
\doclabel{Math F310: Homework 13}
\docdate{Due: November 25, 2016}

\begin{document}

\begin{aproblems}
\vskip 0.5cm

\aproblem Problem 10.5

\aproblem Problem 10.7

\aproblem Continuing with the theme that some sample points are better than others, recall that polynomial interpolation with high-order polynomials is prone to making large oscillation errors, but that this can be minimized
using Chebyshev polynomials, which are the Lagrange polynomials associated
with the sample points 
\[
x_j= \cos(\pi+(\pi j/n)),\qquad j=0,\ldots,n
\]
on the interval $[-1,1]$.
Clenshaw-Curtis integration is integration using polynomial interpolation at
these sample points.

Use the MATLAB \texttt{polyfit} function to perform polynomial
interpolation at these sample points for $n=4,6,10$ and then
use the resulting polynomials to approximate
\[
\int_{-1}^1 x\sin(x)\;dx.
\]
Compare your approximations to the exact answer (which you should
compute by hand).  Integration by parts!

\aproblem Recall that 5 point Gauss-Legendre integration
uses sample points
$[-\beta,-\alpha,0,\alpha,\beta]$ where
\[
\begin{aligned}
\alpha &= \frac{1}{3}\sqrt{5-2\sqrt{\frac{10}{7}}}\\
\beta &= \frac{1}{3}\sqrt{5+2\sqrt{\frac{10}{7}}}.
\end{aligned}
\]
Write a code that performs composite Gauss-Legendre integration
with these sample points.  Your code should have the signature
\begin{verbatim}
function q=glquad(f,a,b,N)
...
end
\end{verbatim}
were $f$ is the function to integrate, $a$ and $b$ are the endpoints
of integration, and $N$ is the number of subintervals. Your code
should perform Gauss-Legendre integration on each subinterval and add them up.
Then apply your function to compute
\[
\int_{-1}^1 x\sin(x)\;dx
\]
using $N=1,2,4,10$.  Compare the results of Gauss-Legendre integration
to the results you saw using Clenshaw-Curtis integration.

\end{aproblems}
\end{document}